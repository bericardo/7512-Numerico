%------------------------------PORTADA---------------------------------%

\documentclass[a4paper,12pt]{article}
\usepackage{latexsym}
\usepackage[spanish]{babel}
\usepackage[utf8]{inputenx}
\usepackage{graphicx}
\usepackage{multirow}
\usepackage{hyperref}
\usepackage{fancyhdr}

\pagestyle{fancy}
\fancyhf{}
\rhead{Nombre del trabajo practico}
\lfoot{\textit{Nombre y Apellido Alumno}}
\rfoot{\textit{Página \thepage}}

\renewcommand{\headrulewidth}{0.4pt} %<- ADDED
\renewcommand{\footrulewidth}{0.4pt} %<- ADDED

\hypersetup{
    colorlinks=true, %set true if you want colored links
    linktoc=all,     %set to all if you want both sections and subsections linked
    linkcolor=black,  %choose some color if you want links to stand out
}

\title{
	\vspace{-3cm}
		{\small Facultad de Ingeniería $\vert$ Universidad de Buenos Aires} \\
		{\small XX Cuatrimestre 20XX} \\
	\rule{400pt}{3pt}\\
		{\Large \textbf{95.10 $\vert$ Modelación numérica} }\\ \vspace{-0.3cm}
		{\large \textbf{75.12 $\vert$ Análisis numérico I A} }\\ \vspace{-0.3cm}
		{\large \textbf{95.13 $\vert$ Métodos matemáticos y numéricos} }
	\vspace{0.2cm}
	\rule{400pt}{3pt}\\
		{\Large \textbf{ Trabajo Práctico \#X} } \\ \vspace{0.5cm}
		{\Large \textbf{Nombre del trabajo practico}}
	\vspace{-2cm}
	\date{}
}


\begin{document}
\maketitle

%---------------------------DATOS DEl ALUMNO---------------------------%

\setlength{\arrayrulewidth}{0.5mm}
\setlength{\tabcolsep}{31pt}
\renewcommand{\arraystretch}{1.5}
\thispagestyle{empty} % Borra numero de pagina de la portada

\begin{center}
\begin{tabular}{ |c|c|c| } 
	\hline
	\multirow{2}{3.2em}{Grupo \centering{Nº}} & Nombre y Apellido Alumno1 & Padron \\ 
	\cline{2 - 3}
	
	& Nombre y Apellido Alumno2 & Padron \\ 
	\hline
\end{tabular}
\end{center}

%--------------------------DATOS DEl DOCENTE---------------------------%

\setlength{\arrayrulewidth}{0.5mm}
\setlength{\tabcolsep}{27pt}
\renewcommand{\arraystretch}{1.5}

\begin{center}
\begin{tabular}{ |c|c|c| } 
	\hline
	\textbf{Fecha} & \textbf{Correcciones/Observaciones} & \textbf{Docente} \\ 
	\hline
	& &  \\
	
	& &  \\
	
	& &  \\
	
	& &  \\
	
	& &  \\
	
	& &  \\
	
	& &  \\
	
	& &  \\
	\hline
\end{tabular}
\end{center}

%--------------------------NOTA FINAL DEL TP---------------------------%

\setlength{\tabcolsep}{37.5pt}
\renewcommand{\arraystretch}{1.8}

\begin{center}
\begin{tabular}{ |c|c|c| } 
	\hline
	\textbf{Calificación Final} & \textbf{Docente} & \textbf{Fecha} \\ 
	\hline
	
	& &  \\
	
	& &  \\
	
	\hline
\end{tabular}
\end{center}

%---------------------------FIN DE LA PORTADA--------------------------%

\newpage
\tableofcontents

\newpage
\section{Introducción}
\paragraph{\normalfont Introducción del problema a resolver.}
\paragraph{\normalfont Comentarios relativos al enunciado.}
\paragraph{\normalfont La realización de un TP apunta a que el alumno logre 
identificar el problema, buscar técnicas de resolución del mismo, 
traducirlas en un código de programación confiable y utilizar éste último 
para extraer resultados de interés. Los resultados deben ser analizados 
y presentados de forma tal que el lector del informe pueda obtener las 
conclusiones importantes en forma rápida y sencilla. En este sentido, 
resulta instructivo que  el alumno haga un paralelo entre el informe de 
un TP y un informe de estilo profesional.Resulta una buena práctica 
ilustrar cuestiones que se plantean en la Introducción.}

\section{Metodología}

\paragraph{\normalfont Métodos a utilizar. Descripción y esquemas numéricos utilizados. Incorporar una descripción gráfica puede ser útil.}
\paragraph{\normalfont Descripción de la codificación.}

\section{Resolución}

\paragraph{\normalfont Informar el desarrollo del trabajo. En esta sección se deberá respetar el orden y numeración de los ítems requeridos en el enunciado. En el caso de no hacerlo se debería ser muy específico con la descripción de las tareas realizadas.}
\paragraph{\normalfont Incorporar tablas y gráficos. }
\paragraph{\normalfont Las tablas deben tener una longitud acotada.}
\paragraph{\normalfont Se deberá ser preciso en la definición de los gráficos. Algunas sugerencias: detallar información de ordenadas y abscisas, incorporar leyenda, tener en cuenta la cantidad de dígitos representativos de los ejes, utilizar todo tipo de líneas y puntos, evaluar incorporar título, manejar tamaños y resolución, exportar a imagen para luego ser importados al texto, etc.}
\paragraph{\normalfont En el Anexo se presenta un código ejemplo para realizar gráficos.}
\paragraph{\normalfont Ejemplo de gráfico:}

\section{Conclusiones}

\paragraph{\normalfont Resaltar los aspectos más relevantes de la ejecución del trabajo práctico.}
\paragraph{\normalfont Concluir acerca de los resultados y los métodos numéricos utilizados.}

\section{Referencias}

\paragraph{\normalfont Skiba, Y., 2005. Métodos y Esquemas Numéricos: Un Análisis Computacional. 1ra. edición. Dirección General de Publicaciones y Fomento Editorial, Universidad Nacional Autónoma de México, México, 440 pp. ISBN: 970-32-2023-1}

\section{Anexo:Códigos}

\paragraph{\normalfont Incorporar los códigos en el ANEXO y explicitar su utilización.}

\subsection{Código de graficación}
\begin{verbatim}
% Rango de cálculo de la variable x
x = -10:0.1:10;

% Lista de elementos a grafica
legend_list = {"sin (x)","cos (x)"};  # cell array

% Graficador
plot (x, sin (x),'r');
hold on;
plot (x, cos (x),'b');
legend (legend_list);
hold off;

% Título
title ("Ploteos - Ejemplos - FIUBA - 95.10");

% Rótulos de ejes
xlabel('x - letra 10','fontsize',10)
ylabel('y - letra 14','fontsize',14) 

% Texto incrustado
text (1.0, 0.75, "Texto ubicado en x=1.0 e y=0.75");
text (-5.0, -0.75, "Texto ubicado en x=-5.0 e y=-0.75");

% Rangos (x_inf x_sup y_inf Y_sup)
axis ([-15 15 -1.5 1.5])

% Grilla
grid

% Tamaño de letra de los n
set(gca,'fontsize',20); % sets font of numbers on axes 

% Exporta el gráfico a un archivo en formato jpg
print -djpg archivo_01.jpg 
\end{verbatim}

\end{document}
